\chapter{关于参考文献}{About references}

只列出作者直接阅读过或在正文中被引用过的文献资料。引用他人成果,在引文前后必须加双引号,并标明序号,在参考文献中列出。
参考文献中先列出直接引用过的资料,再列出直接阅读过且被参考的资料。参考文献要另起一页,一律放在正文之后,不得放在各章节之后。
根据《中国高校自然科学学报编排规范》的要求书写参考文献,并按顺序编码制,作者只写到第三位,余者写“等”。


几种主要参考文献的格式为:
专(译)著:作者.书名(译者).出版地:出版者,出版年,起止页码\\
连续出版物:作者.文题.刊名.年,卷号(期号):起止页码\\
论 文 集:作者.文题.编者.文集名.出版地:出版者,出版年,起止页码\\
学位论文:作者.文题[博士(或硕士)学位论文].授予单位,授予年\cite{许家林1999岩层移动与控制的关键层理论及其应用}\\
技术标准:发布单位.技术标准代号.技术标准名称.出版地:出版者,出版日期\\
英文论文\cite{1978Indexing}


举例如下:
〔例文〕 在出任约翰·霍普金斯大学校长的就职演说中,
吉尔曼阐述了自己的英才主义教育思想:“最慷慨地促进一切有用知识的发展;鼓励研究;
促进青年人的成长,促进那些依靠其能力而献身科学进步的学者的成长”\cite{贺国庆1998德国和美国大学发达史}。 
吉尔曼按照这一思想,在长达25年的校长任期内,把研究生教育放在首位,并全力以赴地发展科学研究,
取得了堪称辉煌的办学成就。据1926年的调查统计,当时每一千位著名的美国科学家中,就有243人是约翰·霍普金斯大学的毕业生\cite{陈树清1982美国研究生教育发展的历程及其特点} 。
参考文献(四号、黑体、顶格)


说明:本模板使用此前现有的bst文件,由于网络上(百度学术,谷歌学术等)得到的bib格式可能会有缺失项,导致输出项目并不完全准确。
