\documentclass[final,MD,numbers,times,AutoFakeBold]{cumtthesis}
% 参数1:文档格式  预印版\终稿\盲审\检查 preprint, final, blindreview, check
%       打印时修改为 final 仅考虑本科生模板,暂时不支持盲审\检查格式
% 参数2:学位  本科\硕士\博士 BD MD PhD
%       仅考虑本科生模板,暂时不支持硕士\博士格式
\usepackage{fontspec}
\usepackage{textcomp, gensymb}
\setmainfont{Times New Roman}
\usepackage{listings}
\usepackage{multirow}
\usepackage{indentfirst}
\usepackage{tikz}
\usepackage{etoolbox}
\usepackage{color}
% 使用更新的伪代码包写法
\usepackage{algorithm}
\usepackage{algpseudocodex}
\usepackage{float}
\usepackage{rotating}
\usepackage{booktabs}
\usepackage{enumerate}
\usepackage{emptypage}
\usepackage[a4paper,left=3.17cm,right=3.17cm,top=2.54cm,bottom=2.54cm]{geometry}
\usetikzlibrary{matrix,calc,shapes,backgrounds,patterns,positioning,decorations.pathreplacing}
%=================================== 数学符号 =================================%
\newcommand{\rtn}{\mathrm{\mathbf{R}}}
\newcommand{\N}{\mathrm{\mathbf{N}}}
\newcommand{\AS}{~\mathrm{a.s.}}
\newcommand*{\PR}{\mathrm{\mathbf{P}}}
\newcommand*{\EX}{\mathrm{\mathbf{E}}}
\newcommand*{\dif}{\,\mathrm{d}}
\newcommand*{\F}{\mathcal{F}}
\newcommand*{\prs}{\dif\PR-\mathrm{a.s.}}
\newcommand*{\pts}{\dif\PR\times\dif t-\mathrm{a.e.}}
\newcommand{\Ito}{It\^{o}}
\newcommand{\tT}[1][0]{[#1,T]}
\newcommand{\intT}[2][T]{\int^{#1}_{#2}}
\newcommand{\s}{\mathcal{S}}
\newcommand{\me}{\mathrm{e}}
\newcommand{\one}[1]{{\bf 1}_{#1}}
\newcommand{\Mm}{{\rm M}}
\newcommand{\circled}[2][]{\tikz[baseline=(char.base)]
    {\node[shape = circle, draw, inner sep = 1pt]
    (char) {\phantom{\ifblank{#1}{#2}{#1}}};%
    \node at (char.center) {\makebox[0pt][c]{#2}};}}
\robustify{\circled}
\setlength{\intextsep}{10.0pt plus 2.0pt minus 2.0pt}

\DeclareMathOperator*{\sgn}{sgn}
%=================================== 数学符号 =================================%
%=================================== Code Style ==============================%
\definecolor{mygreen}{rgb}{0,0.6,0}
\definecolor{mygray}{rgb}{0.5,0.5,0.5}
\definecolor{mymauve}{rgb}{0.58,0,0.82}
\lstset{
    aboveskip=3mm,
    belowskip=3mm,
    showstringspaces=false,
    columns=flexible,
    basicstyle={\normalsize\ttfamily},
    numbers=left,
    frame=L,
    numberstyle=\tiny\color{mygray},
    keywordstyle=\color{blue},
    commentstyle=\color{mygreen},
    stringstyle=\color{mymauve},
    breaklines=true,
    breakatwhitespace=true,
    tabsize=3,
    framextopmargin=2pt,framexbottommargin=2pt,abovecaptionskip=-3pt,belowcaptionskip=3pt,    
    xleftmargin=1em, % 设定listing左右的空白
}

\begin{document}
% 正文前部分
\frontmatter
% 首页\第二页\授权声明\认定书\原创声明
% -------21cm为A4宽度, 3.17cm为标准左右页边距, 0.6,0.9为文本宽度系数

%--------输入中文论文题目,可根据题目适当设置一行宽度,默认(21-3.17*2)*0.6=8.8cm
\CLunWenTiMu[7cm]{基于潜在扩散过程的视频预测模型}

%--------输入英文论文题目,可根据题目适当设置一行宽度,默认(21-3.17*2)*0.9=13.2cm
\ELunWenTiMu[11cm]{A Latent Video Diffudion Model for Video Prediction}

%--------输入论文作者
\ZuoZhe{胡钧耀}

%--------输入导师职称,导师姓名
\DaoShi[副教授]{赵佳琦}
% \DiErDaoShi[讲师]{杨老师}

%--------输入毕业时间
\BiYeShiJian{2023}{6}

%--------输入中图分类号
\ZhongTuFenLeiHao{}

%--------输入UDC
\UDC{}

%--------输入密级
\MiJi{公开}

% --------输入毕业学校, 默认是中国矿业大学
\BiYeXueXiao{中国矿业大学}

%--------输入学校代码, 默认是10290
\XueXiaoDaiMa{10290}

%--------输入学位类别
\XueWeiLeiBie{工学}

%--------输入培养单位
\PeiYangDanWei{计算机学院}

%--------输入学科专业
\XueKeZhuanYe{计算机科学与技术}

%--------输入研究方向
\YanJiuFangXiang{}

%--------输入答辩委员会主席
\DaBianWeiYuanHuiZhuXi{}

%--------输入评阅人
\PingYueRen{}

%--------输入学号
\XueHao{06192081}

\makecover
% 致谢\中文摘要\英文摘要
\input{article/frontmatter}
% 中文目录
\tableofcontents
% 英文目录
\tableofecontents
% 图清单
% \listoffigures
% 表清单
% \listoftables
% 变量注释表
% \begin{notation}[2.5cm]
    \item[$min\_s$] 最小支持度
    \item[$min\_c$] 最小置信度
    \item[$R$] 空间邻近关系
    \item[$row\_instance$] 行实例
    \item[$table\_instance$] 表实例
    \item[$PR(c,f_i)$] 空间参与率
    \item[$PI(c)$] 空间参与度
    \item[$min\_prev$] 最小参与度
    \item[$\delta_i$] 流入量
    \item[$\omega_i$] 流出量
    \item[$\psi_i$] 流入流出比
    \item[$Q$] 模块值
    \item[$\Delta Q$] 模块增益值
\end{notation}
% 正文部分
\mainmatter
% 写新章节后必须在这里include在article文件夹的文件,否则无法编译
\include{article/chap1}
\chapter{浮选柱实验研究}{Experiment Research of Column Flotation}
\section{浮选柱研究现状}{Present Research of Column Flotation}
\begin{algorithm}
    \caption{An algorithm with caption}\label{alg:cap}
    \begin{algorithmic}[1]
        \Require $n \geq 0$
        \Ensure $y = x^n$
        \State $y \gets 1$
        \State $X \gets x$
        \State $N \gets n$
        \While{$N \neq 0$}
        \If{$N$ is even}
            \State $X \gets X \times X$
            \State $N \gets \frac{N}{2}$  \Comment{This is a comment}
        \ElsIf{$N$ is odd}
            \State $y \gets y \times X$
            \State $N \gets N - 1$
        \EndIf
        \EndWhile
    \end{algorithmic}
\end{algorithm}
\include{article/chap3}
\include{article/chap4}
% 正文后部分
\backmatter
% TODO: 参考文献格式.bst和参考文献.bib使用cumt.bst
\bibliographystyle{biblio/cumt}
% \bibliographystyle{biblio/cumt-num}
% \bibliographystyle{biblio/new-cumt-num}
\bibliography{biblio/RefExam}
% 简历
% \include{article/resume}

\GuanJianCi{浮选,旋流,分选机理,浮选动力学,矿物分选}
\BingLieTiMing{Analysis and Data Mining about Spatial Data of Sina Weibo Based on Spatial-Spark}
\LunWenYuZhong{中文}
\XueHao{TS14160005}
\PeiYangDanWeiDaiMa{10290}
\PeiYangDanWeiDiZhi{江苏省徐州市}
\XueZhi{四年}
\LunWenTiJiaoRiQi{2022 年 6 月}
\DaBianWeiYuanHuiChengYuan{王永波,杨化超,陈国良,刘志平}
\LunWenZiZhu{无}
\XueWeiShouYuDanWeiMingCheng{}
\XueWeiShouYuDanWeiDaiMa{}
\XueWeiJiBie{}
\LunWenTiMing{}
\PeiYangDanWeiMingCheng{}
\YouBian{}
\XueWeiShouYuNian{}

\makebackcover
\printindex
\clearpage
\end{document}
